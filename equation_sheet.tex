\documentclass{article}

\usepackage{amsmath}
\usepackage[utf8]{inputenc}
\usepackage[english]{babel}
\usepackage{multicol}
\usepackage[margin=0.5in]{geometry}

\setlength{\parindent}{0em}
\setlength{\parskip}{1em}

\newcommand{\var}{\text{Var}}

\begin{document}
\begin{multicols*}{2}

\begin{enumerate}
    \item $A \cup (B \cap C) = (A \cup B) \cap (A \cup C)$
    \item $A \cap (B \cup C) = (A \cap B) \cup (A \cap C)$
\end{enumerate}

\begin{enumerate}
    \item $(A \cap B)' = A' \cup B'$
    \item $(A \cup B)' = A' \cap B'$
\end{enumerate}

\begin{align*}
    {}_n P_r = \frac{n!}{(n-r)!}
\end{align*}

\begin{align*}
    \begin{pmatrix}
        n \\
        r
    \end{pmatrix} = \frac{n!}{r! (n-r)!}
\end{align*}

\begin{align*}
    P(A|B) = \frac{P(A \cap B)}{P(B)}
\end{align*}

\begin{align*}
    P(A \cap B) = P(A|B) P(B) = P(B|A) P(A)
\end{align*}

\begin{align*}
    P(A_j|B) \frac{P(A_j \cap B)}{P(B)} = \frac{P(B|A_j) P(A_j)}{\sum_{i = 1}^k P(B|A_i) P(A_i)}
\end{align*}

\subsubsection{Mean}
\begin{align*}
    \mu = \sum x \times p(x)
\end{align*}

\begin{align*}
    \mu = E(X) &= \sum_{x \in D} x \times f(x) \\
    E(u(X)) &= \sum_{x \in D} u(x) \times f(x) \\
    &= E(c_1 u_1(X) + c_2 u_2(X)) \\
    &= c_1 E(u_1(X)) + c_2 E(u_2(X))
\end{align*}

\subsection{Special mathematical expectations}

\begin{align*}
    \sigma^2 &= E((X - \mu)^2) \\
    &= \sum_{x \in D} (x - \mu)^2 \times f(x) & \text{Definition formula} \\
    &= E(X^2) - \mu^2 & \text{Computation formula}
\end{align*}

\begin{center}
    \begin{tabular}{|c|c|c|}
        \hline
        & Variance & Standard deviation \\ \hline
        Population & $\sigma^2$ & $\sigma$ \\ \hline
        Sample & $s^2$ & $s$ \\ \hline
    \end{tabular}
\end{center}

\begin{align*}
    s^2 &= \frac{1}{n - 1} \sum_{i = 1}^n (x_i - \bar{x})^2 \\
    &= \frac{1}{n - 1} \sum_{i = 1}^n (x_i^2 - n\bar{x}^2)
\end{align*}

\begin{align*}
    \var(cX) &= c^2 \var(X) \\
    \var(X + c) &= \var(X)
\end{align*}

\subsection{Bernoulli distribution}

Used for experiments where there are two outcomes.

\begin{align*}
    f(x) &= p^x (1-p)^{1-x} \\
    E(X) &= p \\
    \var(X) &= p(1-p)
\end{align*}

\subsection{Binomial distribution}

\begin{align*}
    f(x) &= \begin{pmatrix}
        n \\
        x
    \end{pmatrix} p^x (1-p)^{n-x} \\
    E(X) &= np \\
    \var(X) &= np(1-p) \\
    \text{Stdev}(X) &= \sqrt{np(1-p)}
\end{align*}

\subsection{Poisson distribution}

\begin{align*}
    f(x) &= \frac{e^{-\lambda} \lambda^x}{x!} \\
    E(X) &= \lambda \\
    \var(X) &= \lambda
\end{align*}

The poisson distribution can be used to approximate a binomial distribution when $n$ is large and $p$ is small.

\end{multicols*}
\end{document}