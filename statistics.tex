\documentclass{article}

\usepackage{amsmath}
\usepackage[utf8]{inputenc}
\usepackage[english]{babel}
\usepackage{multicol}
\usepackage[margin=1in]{geometry}

% For math fonts, if needed
% \usepackage{amssymb}

\setlength{\parindent}{0em}
\setlength{\parskip}{1em}

\title{Probability and Statistics Notes}
\author{Tim Harding}
\date{Autumn 2021}

\begin{document}
\begin{multicols*}{2}

\section{Probability}

\subsection{Properties of Probability}

\textbf{Random experiments}: Experiments where the outcome is uncertain.

\textbf{Outcome space}: The set of all possible outcomes $S$.

\textbf{Event}: A set of outcomes $A$ such that $A \in S$. The event $A$ has \textit{occurred} when some outcome of a random experiment $a$ occurs where $a \in A$.

\textbf{Mutually exclusive events}: $A$ and $B$ are mutually exclusive if $A \cap B = \emptyset$

\textbf{Exhaustive events}: The set $A_1, A_2, \ldots, A_k$ is exhaustive if $A_1 \cup A_2 \cup \ldots \cup A_k = S$

\textbf{Frequency}: $N(A)$ is the number of times $A$ occurred in $n$ repetitions. 

\textbf{Relative frequency}: $\frac{N(A)}{n}$

\textbf{Probability}: $P(A)$ is the relative frequency of $A$ as $n$ grows arbitrarily large. It satisfies the following properties for the mutually exclusive events $A_1, A_2, \ldots, A_k$ where $k$ may be infinite:
\begin{enumerate}
    \item $P(A_n) \geq 0$
    \item $P(S) = 1$
    \item $P(A_1 \cup A_2 \cup \ldots \cup A_k) = \sum_{n=1}^k{P(A_n)}$
\end{enumerate}

Also note that
\[P(A) = \frac{N(A)}{N(S)}\]
when discussing permutations and combinations. 

\subsection{Methods of Enumeration}

\textbf{Multiplication Principle}: Given experiments $E_1$ with $n_1$ possible outcomes and $E_2$ with $n_2$ possible outcomes, the composite experiment $E_1 E_2$ has $n_1 n_2$ possible outcomes. 

\textbf{Permutation}: An arrangement of $n$ objects. $n!$ arrangements are possible. 

\textbf{Ordered sample}: An ordered sample of $r$ objects taken from a set of $n$ objects. 

\textbf{Sampling with replacement}: When a given object may be selected multiple times in a sample. There are $n^r$ ordered samples with replacement.

\textbf{Sampling without replacement}: Each object may be selected only once in a sample. 

\textbf{Permutations of $\mathbf{n}$ taken $\mathbf{r}$}: 
The number of permutations for $n$ objects filling $r$ positions.
\[{}_n P_r = \frac{n!}{(n-r)!}\]

There are ${}_n P_r$ ordered samples without replacement.

\textbf{Combinations of $\mathbf{n}$ taken $\mathbf{r}$}: The number of unordered subsets of size $r$ that can be selected from $n$ objects where $r \leq n$. This is pronounced \textit{n choose r}.
\[{}_n C_r = \begin{pmatrix}
    n \\
    r
\end{pmatrix} = \frac{n!}{r! (n-r)!} = \frac{{}_n P_r}{r!}\]

There are ${}_n C_r$ unordered samples without replacement.

\textbf{Binomial coefficients}: The numbers $n$ and $r$.

\textbf{Distinguishable permutation}: One of the ${}_n C_r$ permutations of $n$ objects. 

\end{multicols*}
\end{document}